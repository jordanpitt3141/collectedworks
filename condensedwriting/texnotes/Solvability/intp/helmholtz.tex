\documentclass[12pt]{article}
\usepackage{amsmath}
\usepackage{ amssymb }
\usepackage{breqn}
\begin{document}

\title{SWW soliton}
\author{Jordan Pitt / u5013521}

\section{Helmholtz Decomposition}
We know in 2D the equation for the conserved quantity $\vec{G}$ in terms of the conserved quantity $h$ and the primitive variables $\vec{u}$ is

\begin{equation}
\label{Geq}
\vec{G} = h\vec{u} - \nabla\left(\frac{h^3}{3} \left(\nabla \cdot \vec{u}\right)\right)
\end{equation}

To look at this we consider the Helmholtz decomposition of $\vec{G}$ (assuming sufficient smoothness), in principle we can calculate these two parts since we know $\vec{G}$. Thus we have
\[\vec{G} = \nabla \phi + \nabla \times \vec{A}\] 

Taking the curl of \eqref{Geq}

\[\nabla \times \vec{G} = \nabla \times\left(h\vec{u} \right)-\nabla \times \nabla\left(\frac{h^3}{3} \left(\nabla \cdot \vec{u}\right)\right)\]

since the curl of a grad is 0

\[\nabla \times \vec{G} = \nabla \times\left(h\vec{u} \right)\]

so the curl of $\vec{G}$ and $h\vec{u}$ are the same. Since the other term is a gradient, its helmholtz decomposition has no curl part and so it must be that for the helmholtz decomposition of $h\vec{u}$ (which we do not know) the curl part is the same as for $\vec{G}$ so that

\[h\vec{u} = \nabla \psi + \nabla \times \vec{A}\]

So we have that:

\[\nabla \phi + \nabla \times \vec{A} = \nabla \psi + \nabla \times \vec{A} - \nabla\left(\frac{h^3}{3} \left(\nabla \cdot \vec{u}\right)\right)\]

\[\nabla \phi = \nabla \psi - \nabla\left(\frac{h^3}{3} \left(\nabla \cdot \vec{u}\right)\right)\]

Assuming we can just integrate without a problem (probably need to impose some boundary conditions here)

\[ \phi = \psi - \left(\frac{h^3}{3} \left(\nabla \cdot \vec{u}\right)\right)\]

From above by diving the Helmholtz decomposition of $\vec{u}$ by h we get (asssume $h > 0$ )

\[\vec{u} = \frac{\nabla \psi }{h} + \frac{\nabla \times \vec{A} }{h}\]

Subbing this in gives

\[ \phi = \psi - \left(\frac{h^3}{3} \left(\nabla \cdot \left(\frac{\nabla \psi }{h} + \frac{\nabla \times \vec{A} }{h}\right)\right)\right)\]

Getting ready for a integration by parts we divide out the $h^3/3$ factor to give

\[ \frac{3}{h^3}\phi = \frac{3}{h^3}\psi - \nabla \cdot \left(\frac{\nabla \psi }{h} + \frac{\nabla \times \vec{A} }{h}\right)\]

So integrating over the domain and multiplying by a test function $v \in C^{\infty}_0 (\Omega)$

\[ \int_{\Omega} \frac{3}{h^3}\phi v \; dx = \int_{\Omega} \frac{3}{h^3}\psi v \; dx - \int_{\Omega}\nabla \cdot \left(\frac{\nabla \psi }{h} + \frac{\nabla \times \vec{A} }{h}\right) v \; dx\]

By integrating by parts and remembering that v has is supported inside $\Omega$ we have

\[ \int_{\Omega} \frac{3}{h^3}\phi v \; dx = \int_{\Omega} \frac{3}{h^3}\psi v \; dx - \int_{\Omega}\left(\frac{\nabla \psi }{h} + \frac{\nabla \times \vec{A} }{h}\right) \nabla \cdot v \; dx\]

\[ \int_{\Omega} \frac{3}{h^3}\phi v \; dx = \int_{\Omega} \frac{3}{h^3}\psi v \; dx - \int_{\Omega} \frac{\nabla \psi }{h}\left(\nabla \cdot v\right) + \frac{\nabla \times \vec{A} }{h}\left(\nabla \cdot v\right) \; dx\]

\[ \int_{\Omega} \frac{3}{h^3}\phi v - \frac{\nabla \times \vec{A} }{h}\left(\nabla \cdot v\right)  \; dx = \int_{\Omega} \frac{3}{h^3}\psi v \; dx - \int_{\Omega} \frac{\nabla \psi }{h}\left(\nabla \cdot v\right)  \; dx\] \\ \\ \\






In principle we can calculate the LHS for any $v$ in the following way. 

We know $\vec{G}$ and $h$, by definition $\phi$ and $\vec{A}$ are given by

\[\vec{G} = \nabla \phi + \nabla \times \vec{A}\] 

Note that we need $\phi$ and only $\nabla \times \vec{A}$, since

\[\nabla \cdot \vec{G} =  \nabla \cdot (\nabla \phi) +  \nabla \cdot(\nabla \times \vec{A})\] 

\[\nabla \cdot \vec{G} =  \nabla \cdot (\nabla \phi)\] 

multiplying by $v \in C^\infty_0(\Omega)$ and integrating over $\Omega$

\[\int_{\Omega}\nabla \cdot \vec{G} v \; dx =  \int_{\Omega}\nabla \cdot (\nabla \phi) v \; dx\] 

Integrating by parts gives:

\[\int_{\Omega}\vec{G}  (\nabla \cdot v) \; dx =  \int_{\Omega}(\nabla \phi)  (\nabla \cdot v) \; dx\]

 














\end{document}
