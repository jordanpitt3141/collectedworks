\documentclass[12pt]{article}
\usepackage{amsmath}
\usepackage{ amssymb }
\usepackage{breqn}
\begin{document}

\title{SWW soliton}
\author{Jordan Pitt / u5013521}

\section{Helmholtz Decomposition}
We know in 2D the equation for the conserved quantity $\vec{G}$ in terms of the conserved quantity $h$ and the primitive variables $\vec{u}$ is

\begin{equation}
\label{Geq}
\vec{G} = h\vec{u} - \nabla\left(\frac{h^3}{3} \left(\nabla \cdot \vec{u}\right)\right)
\end{equation}

To look at this we consider the Helmholtz decomposition of $\vec{G}$ (assuming sufficient smoothness), in principle we can calculate these two parts since we know $\vec{G}$. Thus we have
\[\vec{G} = \nabla \phi + \nabla \times \vec{A}\] 

Taking the curl of \eqref{Geq}

\[\nabla \times \vec{G} = \nabla \times\left(h\vec{u} \right)-\nabla \times \nabla\left(\frac{h^3}{3} \left(\nabla \cdot \vec{u}\right)\right)\]

since the curl of a grad is 0

\[\nabla \times \vec{G} = \nabla \times\left(h\vec{u} \right)\]

so the curl of $\vec{G}$ and $h\vec{u}$ are the same. Since the other term is a gradient, its helmholtz decomposition has no curl part and so it must be that for the helmholtz decomposition of $h\vec{u}$ (which we do not know) the curl part is the same as for $\vec{G}$ so that

\[h\vec{u} = \nabla \psi + \nabla \times \vec{A}\]

So we have that:

\[\nabla \phi + \nabla \times \vec{A} = \nabla \psi + \nabla \times \vec{A} - \nabla\left(\frac{h^3}{3} \left(\nabla \cdot \vec{u}\right)\right)\]

\[\nabla \phi = \nabla \psi - \nabla\left(\frac{h^3}{3} \left(\nabla \cdot \vec{u}\right)\right)\]

Assuming we can integrate(seems reasonable) we get that (and dropping the constant)

\[\phi = \psi - \left(\frac{h^3}{3} \left(\nabla \cdot \vec{u}\right)\right)\]

It would be nice to rewrite the divergence of $\vec{u}$ in terms of $\psi$ since 
\[h\vec{u} = \nabla \psi + \nabla \times \vec{A}\]
\[\vec{u} = \frac{\nabla \psi}{h} + \frac{\nabla \times \vec{A}}{h} \]

substituting this in gives

\[\phi = \psi - \left(\frac{h^3}{3} \left(\nabla \cdot \left[h^{-1} \nabla \psi + h^{-1}\nabla \times \vec{A}\right]\right)\right)\]

\[\phi = \psi - \left(\frac{h^3}{3} \left(h^{-1} \left(\nabla \cdot \nabla \psi\right) + \nabla \psi \cdot \nabla h^{-1} + \left(\nabla \times A\right)\cdot\nabla h^{-1}\right)\right)\]

Since 
\[\nabla h^{-1} = -h^{-2}\nabla h\]

\[\phi = \psi - \left(\frac{1}{3} \left(h^2 \nabla^2 \psi - h\nabla \psi \cdot \nabla h -h \left(\nabla \times A\right)\cdot\nabla h\right)\right)\]

\[\phi = \psi - \frac{h^2}{3} \nabla^2 \psi - \frac{h}{3}\nabla \psi \cdot \nabla h - \frac{h}{3} \left(\nabla \times A\right)\cdot\nabla h\]

\[\psi - \frac{h^2}{3} \nabla^2 \psi - \frac{h}{3}\nabla \psi \cdot \nabla h = \phi + \frac{h}{3} \left(\nabla \times A\right)\cdot\nabla h\]

Where only the LHS has unknowns. So this is solvable (meaningful?) if we are allowed to firstly take the helmholtz decomposition of both $\vec{G}$ and $h\vec{u}$. We also integrate all the gradient terms. We must also assume that $h > 0$ to do the divisions and finally take the gradient of $h$. Remember that we have dropped a constant as well during integration.  

To do such a thing we must assume that at least
\begin{itemize}
\item{$\Omega$ is the boundary of the problem to do a helmholtz decomposition we must have that $\Omega$ is a bounded, simply-connected, Lipschitz domain}
\item{$\vec{G}$ and $h\vec{u}$ must be $L^2\left(\Omega\right)^3$ function to do a helmholtz decomposition. The resultant decomposition as the base divergence free part is in $H(\text{curl},\Omega)$ while the base of curl free part is in $H^1\left(\Omega\right)$.}
\end{itemize}

By these equations we start with:

$G \in L^2$ and $h \in H^1$ then $\phi \in H^1$ and $\nabla \times A \in L^2$. Also $\nabla h \in L^2$, so we have that $\psi \in H^2$ thus $\vec{u} \in L^2$











\end{document}
